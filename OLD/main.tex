\newqed\documentclass{article}
%color 使えない \colorlet{grayfz}{gray!40}
% line numbers
%\usepackage{mathptmx}
\usepackage[11pt]{moresize}
\renewcommand{\baselinestretch}{1.2}
\usepackage[utf8]{inputenc}
\usepackage{url}

\usepackage[utf8]{inputenc}
\usepackage[english]{babel}

\usepackage{lipsum}
\usepackage{xcolor}
\usepackage{tikz}
\usepackage{mathtools,amsfonts,amssymb,amsthm}
\usepackage{graphicx}
\definecolor{ocre}{RGB}{52,177,201}
\definecolor{grayfz}{RGB}{210,200,190}

%lineno
\usepackage[right]{lineno}
\renewcommand{\linenumberfont}{\normalfont\bfseries\small\color{grayfz}}
%theorem
\theoremstyle{definition}
\newtheorem{definition}{Definition}[section]
\newtheorem{notation}{Notation rule}[section]
 \newtheorem{proposition}{Proposition}
 \newtheorem{lemma}{Lemma}
 \newtheorem{corollary}{Corollary}
  \newtheorem{conjecture}{Conjecture}[section]
  \newtheorem{axiom}{Axiom}[section]
  
\newcommand{\newspc}{\textcolor{white}{s}}
\newcommand{\newspace}{\textcolor{white}{1234567890}}
\newcommand{\newqed}{\newspc{\hfill\color{ocre}\ensuremath{\blacksquare}}}
\newcommand{\newdef}{_{def}}
\newcommand{\newland}{\newspc_{and}$\land$\newspc}
\newcommand{\newvee}{\newspc_{or}$\vee$\newspc}
\newcommand{\newset}{\newspc{$\overset{\mathrm{\{\}}}{..}$}}
\newcommand{\newcmnt}{\newspc in\_terms\_of \newspc}
\newcommand{\newset}{\newspc \#set\#def \newspc}
\newcommand{\newsub}{\newspc \#subset\#set\#where \newspc}
\newcommand{\newfunc}{\newspc \#function\#def \newspc}

 
\title{Order consistent logic}
\author{Shigeo Hattori}
\date{\today
\\{\small First version: November, 2019}
\\{\small bayship.org@gmail.com}
\\{\small \url{https://github.com/bayship-org/mathematics}}
}

\usepackage{natbib}
\usepackage{graphicx}

\begin{document}

\maketitle
\section{Prerequisite definitions and notations }
\\\url{https://github.com/bayship-org/mathematics/blob/master/Minor_of_memBer.pdf}
\\\newspc
\begin{itemize}
\end{itemize}
\linenumbers
In the article above, in its first two pages, all prerequisite definitions for this article are given. Especially, the second page of the above article defines that two memBers $(x,y)$ are said ($x$ is a \textbf{minor} of $y$).


\begin{definition}
\newspc\\"And" is also written as "\newland".
\\"Or" is also written as "\newvee".
\newqed

\end{definition}

\section{Introduction}
\\This article defines new words, \textbf{"a logical expression is order consistent"}, and gives a conjecture for the new words and the notion of minors of memBers. In the rest of this introduction, the main conjecture is roughly introduced in a style of an example.
\\\newspc
\\Let $(X,T^2)$ be the Euclidean space of 2-dimension.
\\Let $x1$ be some closed line segment in terms of some coordinate system.
\\Let $C:=\{x$ $|$ $((T^2,x),(T^2,x1))$ are isomorphic $\}$.
\\Let $f$ be a function on $C$ as $f(x)=$length$(x)$.
\\Take $\foray y:\in$ image$(f)$,
\\Let $k:=\{x$ $|$ $f(x)=y\}$.
\\Then the conjecture claim that
\\if the definition of $k$ is order consistent
\\then $k$ is a minor of $(X,T^2)$.
\newqed

\section{Definitions}



\begin{definition}[Order consistent]
\newspc\\Take $\forall L$ as a logical expression such that *1 holds. 
\\Then $L$ is said order consistent.
\begin{description}
\item [1.] Take $\forall (f,p,q,r,d1,d2)$ 
\\such that (*2 \newland ..... \newland *9) holds
\\then (*10 \newland *11) holds. 
\item [2.] $L$ defines $f$.
\item[3.] $f$ is a function. 
\item[4.] $\{p,q,r\}\subset$ domain$(f)$.
\item[5.] $(d1,d2)$ are total orders on $\{p,q,r\}$.
\item [6.] Take $\forall d:\in \{d1,d2\}$.
\item[7.] Take $\forall (s,t):\in d$.
\item[8.] There exists $\exists x:\in$domain$(f)$.
\item[9.] Evaluation of $f(x)$ refers to $(s,t)$.
\item [10.] $d1=d2$.
\item [11.]If $p< q< r$ 
\\then $f(p)\leq f(q) \leq f(r) \vee  f(r)\leq f(q) \leq f(p)$.
\newqed
\end{description}
\end{definition}

\section{Main conjecture}

\begin{conjecture}[Main conjecture]

\newspc \\Let $(n,X,T^n)$ be the Euclidean space of $n$-dimension.
\\Take $\forall (L,f,k)$ such that (*1  \newland ..... \newland *3).
\\Then $k$ is a minor of $(X,T^n)$.
\begin{description}
\item[1.] $L$ is an order consistent logical expression.
\item[2.] $L$ takes $(n,X,T^n)$ as the antecedent.
\item[3.] $L$ as a consequent specifies a function $f$ relatively to the antecedent,$(n,X,T^n)$.
\item[4] There exists $\exists y:\in$ image$(f)$. Then $k=\{x\in$ domain$(f)$ $|$ $f(x)=y\}$.
\end{description}
\newqed
\end{conjecture}
\\
\section{Examples}
This section just gives examples of substituting actual values into variables of the antecedent of the main conjecture.
\\
\begin{definition}[Unknot]
\newspc\\For the main conjecture,this example substitutes values 
\\into $L$ as (*1 \newland ..... \newland *10).
\begin{description}
\item[1.] $n$:=3.
\item[2.] $(n,X,T^n)$ is the Euclidean space $(X,T^n)$ of $n$-dimension.
\item[3.] Let $x1$ be an unknot.
\item[4.] Let $C:=\{x$ $|$ $((T^n,x),(T^n,x1))$ are isomorphic $\}$.
\item[5.] Take $\forall x:\in C$.
\item[6.] $f1(x):=\{d$ $|$ $d$ is a proper knot diagram of $x\}$.
\item[7.] $f2(x):=\{r$ $|$ 
\\$\exists d:\in f1(x)$ \newland 
\\$r$ is the number of crossings on $d$
\\$\}$.
\item[8.] $f(x)$ returns the maximum number of $f2(x)$.
\item[19.] Take $\forall y:\in$ image$(f)$. 
\item[10.] \footnote{For example, you can specify $y$ as $y:=10$.}
Let $k:=\{x\in$ domain$(f)$ $|$ $f(x)=y \}$
\end{description}
\newspc\newqed
\end{definition}
\\
\begin{proposition}[Non order consistent]
\newspc\\Refer to $f$ of the previous definition. 
\\Let $g(x):=(f(x)-2)^2$. 
\\Then the definition of $g$ is not order consistent. 
\newspc\newqed
\end{proposition}
\begin{proof}
\newspc
\begin{itemize}
    \item Refer to the previous definition for $f2$.
    \item The definition of $g$ is dependent on $f2$ and the standard order on image($f2$).
    \item By the way, let $(p,q,r):=(1,2,3)$. 
    \item Then $p \textless q \textless r$ \newspc $\land$ \newspc $g(q) \textless g(p)=g(r)$.
    
\end{itemize}
\end{proof}
\\

\pagebreak
\\
\section{Appendix-Specification}
\\In mathematics, to specify an entity is always relative to the context. For example, let the antecedent take $\forall (n,X,T,M)$ as a Euclidean space $(X,Y,M)$ of  $n$ dimension then as the consequent you can not specify any single point of the space. Contrary, if the antecedent has taken a coordinate system $C$, the consequent can specify any point of the space relatively to the antecedent. 
\\Although the Euclidean space $(X,T,M)$ is not specified the specific dimension, $(X,T,M)$ is said specified for the consequent. But $(X,T,M)$ is not said specified for the antecedent.

\begin{definition}
\\For all variable $x$ of the consequent, if exactly one entity of the antecedent can be substituted into $x$ then $x$ is said having been specified relatively to the antecedent. 
\newqed
\end{definition}




\bibliographystyle{plain}
\bibliography{references}
\end{document}
